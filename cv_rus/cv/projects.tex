%-------------------------------------------------------------------------------
%	SECTION TITLE
%-------------------------------------------------------------------------------
\cvsection{Университетские проекты}


%-------------------------------------------------------------------------------
%	CONTENT
%-------------------------------------------------------------------------------
\begin{cventries}

%---------------------------------------------------------
  \cventry
    {СибГУТИ} % Organisation
    {Беспроводной протокол передачи данных на базе SDR} % Project
    {Новосибирск, Россия} % Location
    {Май 2019 - Июль 2020} % Date(s)
    {
      \begin{cvitems} % Description(s) of project
        \item {Разработка простого радио протокола на базе HackRf One}
        \item {Глубокое изучение и доработка hardware/software частей HackRF One}
        \item {Обновление firmeware для изменения hardware обработки данных на HackRF One}
        \item {Использование Intel Cyclone 10 FPGA для обработки радио данных с модулями синхронизации несущей/бит/сэмплов}
        \item {Использование BeagleBone PRU ядер для обработки пакетов с/на HackRF One через дополнительный FPGA}
        \item {\textbf{Технические навыки:} C/C++, CrossBuilding, Microcomputers, MakeFile, Verilog, MatLab, Quartus, Git}
        \item {\textbf{Социальные навыки:} Time Management, работа в команде, презентации, отчеты}
      \end{cvitems}
    }

%---------------------------------------------------------
  \cventry
    {СибГУТИ, Hochschule Anhalt} % Organisation
    {Плата SDR (Software defined radio) для тестирование беспроводных протоколов передачи данных} % Project
    {Новосибирск, Россия\newline Köthen, Germany} % Location
    {Июль 2020 - Апрель 2022} % Date(s)
    {
      \begin{cvitems} % Description(s) of project
        \item {Разработка SDR платы на базе FPGA}
        \item {Расчет hardware НЧ фильтров}
        \item {Разработка радио части платы на базе IQ модулятора/демодулятора и радио смесителя}
        \item {Разработка цифровой части платы на базе FPGA, Ethernet10/100/1000, USB, GPIO}
        \item {Производство платы и первичное тестирование}
        \item {\textbf{Технические навыки:} SystemVerilog, Hardware debugging, Altium Designer, Quartus, Questa, Git}
        \item {\textbf{Социальные навыки:} презентации, руководство, работа в команде, логическое мышление}
      \end{cvitems}
    }

%---------------------------------------------------------
\end{cventries}
